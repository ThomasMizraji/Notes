% !TEX root = ../notes.tex

% ================  (Interactive) Visualization ==============

\section{Data Visualization}


\subsection{Two main purposes}

In this chapter we are going to talk about visualization in general and then about what it is called interact visualization. The first thing we do when we talk about visualization is to split it into two different tracks. 

\begin{itemize}
\item {The first one is about \emph{analysis}: you want to support reasoning about information. For instance, when you have a \texttt{DataFrame} you can make a plot of the distribution of the attribute in order to identify outliers, missing data and \emph{so forth}. In general with this kind of visualization you can do more like discover structures, quantify values and influences.
\emph{This way of using visualization is extremely important for debugging purposes}.}

\item{The second  is the \emph{communication} part and it is about informing and persuade people.  The key difference in working in \emph{Data Science} and exclusively in \emph{Machine Learning} or \emph{Statistics} is the fact that you don't just stop after getting a good model and evaluating its accuracy. You would make a story that you convey to people.
For this reason we have to use visualization that can capture attention, can engage people and can tell a story visually (tell a story using visual tools takes a lot less time) and at last but not least, you are focused only on certain aspects omitting others. It is a double edge sword. That is because, on one hand you have to consider that there is an information overflow that people are suffering in general (we get to many media in which we consume information) so we do not want our visualization conveys more information than  a human being can actually get in a few seconds). On the other hand we have to do it carefully, avoiding to omit some information just because difficult to handle or with, apparently, non sense.}
\end{itemize}

\subsection{Data exploration}

In order to do a good \emph{Data exploration} analysis by mean of visualization:

\begin{itemize}
\item {Get familiar with your favorite graphing package:}
\begin{itemize}
\item \texttt{Matplotlib} which is widely used in \texttt{Python}
\item \texttt{Seaborn} and \texttt{Bokeh} that are two additions on top of \texttt{Matplotlib}
\item \texttt{D3.js} (\texttt{Javascript}) is the most famous framework for interactive graphics
\end{itemize}
\item {Get fluent with plotting:}
\begin{itemize}
\item Histograms
\item Scatter plots
\item Line and bar plots
\end{itemize}
\end{itemize} 

\subsubsection{One variable}

Whenever we want to look at the data we can use histograms, they tell us a lot about the single variable. Once you plot them you can try to figure out their distribution, for instance we can identify skewed distributions, multimodal or long tail data. The latter is characterized by a bunch of bins that reveal a lot of occurrences and bars in the tail where we observe a very few occurrences. Many of this long tail data follow a \href{https://en.wikipedia.org/wiki/Power\_law\#Power-law\_probability\_distributions}{\emph{power-law}}. To claim the latter we need to run some test on data that proofs the statistical significance of our hypothesis, otherwise we can just state that it looks like a \emph{power-law}. For a graphical representation:

\begin{enumerate}
\item Sort the histogram counts by magnitude, descending.
\item Plot count vs bucket number on a log-log plot.
\end{enumerate}


INSERT FIG, the Zipf and preferential attachment links


Generally this law is characteristic of social-influence processes, to know more look up for \emph{Preferential attachment}.

The \emph{multinomial} data registers more than one peak in the histogram, it suggests that there are two or more distinct populations of a sample. When you deal with something like this do not guess! Explore further by using, e.g., color and a histogram of multiple populations. 


INSERT FIG


Sometimes data is weird and is very hard to explain. Also in this case, do not guess! Trace through the data pipeline to find where the strangeness comes from. Usually it is a processing bug. Hence, check your code!

There is a way for a \emph{proactive Weird data Detection}. If data looks normal, take a picture and save it for later, then periodically compare new data with old whenever there is a pipeline update. Generally always try to have a theory of what the data should look like!

\subsubsection{More than one variable}
 
Most of the time we are interested in visualizing more than one variable, here a \emph{non-thorough} list of possibility is listed:


 INSERT FIG FOR EACH OF THEM


\begin{itemize}
\item Two variables 
\begin{itemize}
\item \emph{Scatter plots} quickly expose the relationships between two variables
\end{itemize} 
\item  More than two variables
\begin{itemize}
\item \emph{Stacked plot}: stack variable is discrete, useful to explore data
\item \emph{Parallel coordinate plot}: one discrete variable, an arbitrary number of other variables (when this number increases it risks to become very messy)
\item \emph{Radar Chart}: one discrete variable (through the radar design), an arbitrary number of other variables
\end{itemize}
\end{itemize}





When you deal with an high number of variables, a valid idea to visualize in a better way is to reduce the number of variables applying algorithms, this pocess is called \emph{Dimensionality reduction}, one example is the \emph{PCA} [ADD link PCA]. Intuitively, given twenty different variables, many tend to not variate a lot, \emph{PCA} extracts the couple of attributes that really do the difference allowing visualization of high-dimensional continuous data in 2D using principal components. Hence, instead of directly plot multivariate data, try to think whether a dimensionality reduction can be useful.

We argued for analysts is important to form expectations of what the data should look like. This helps against pipeline errors and to identify interesting patterns. But beware of seeing \emph{Martian Canals}: do not see things that are not there. Moreover, an observer should also be attuned to patterns that are not part of his theory, in other words to expect the unexpected. 


\subsection{Moving towards Interactive viz}

Interactive visualization is a new field and it's getting more and more common. Our aim is to deliver results and this has been enabled by the new web technologies and in general by few frameworks essential for the current state of the art. \texttt{JavaScript} plays a very important role in the field. The vast majority of the libraries that allow to do visualization are in \href{https://www.codecademy.com/learn/javascript}{\texttt{JavaScript}}, so if you know how to use it or if you want to learn how to use it, it is definitely a good tool to have in your toolbar. 



%Visualize is faster than read

%Future journalism


%Anti-Examples

%Visualization definitions

%10 rules

%There rules hold for the interactive data as well.

